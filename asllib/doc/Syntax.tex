%%%%%%%%%%%%%%%%%%%%%%%%%%%%%%%%%%%%%%%%%%%%%%%%%%%%%%%%%%%%%%%%%%%%%%%%%%%%%%%%
\chapter{Syntax\label{chap:Syntax}}
%%%%%%%%%%%%%%%%%%%%%%%%%%%%%%%%%%%%%%%%%%%%%%%%%%%%%%%%%%%%%%%%%%%%%%%%%%%%%%%%

This chapter defines the grammar of ASL. The grammar is presented via two extensions
to context-free grammars --- \emph{inlined derivations} and \emph{parametric productions},
inspired by the Menhir Parser Generator~\cite{MenhirManual} for the OCaml language.
Those extensions can be viewed as macros over context-free grammars, which can be
expanded to yield a standard context-free grammar.

Our definition of the grammar and description of the parsing mechanism heavily relies
on the theory of parsing via LR(1) grammars and LR(1) parser generators.
%
See ``Compilers: Principles, Techniques, and Tools''~\cite{ASU86} for a detailed
definition of LR(1) grammars and parser construction.

The expanded context-free grammar is an LR(1) grammar, modulo shift-reduce
conflicts that are resolved via appropriate precedence definitions.
That is, given a list of token, returned from $\aslscan$, it is possible to apply
an LR(1) parser to obtain a parse tree if the list of tokens is in the formal language
of the grammar and return a parse error otherwise.

The outline of this chapter is as follows:
\begin{itemize}
  \item Definition of inlined derivations (see \secref{InlinedDerivations})
  \item Definition of parametric productions (see \secref{ParametricProductions})
  \item ASL Parametric Productions (see \secref{ASLParametricProductions})
  \item Definition of the ASL grammar (see \secref{ASLGrammar})
  \item Definition of parse trees (see \secref{ParseTrees})
  \item Definition of priority and associativity of operators (see \secref{PriorityAndAssociativity})
\end{itemize}

\section{Inlined Derivations \label{sec:InlinedDerivations}}
Context-free grammars consist of a list of \emph{derivations} $N \derives S^*$
where $N$ is a non-terminal symbol and $S$ is a list of non-terminal symbols and terminal symbols,
which correspond to tokens.
We refer to a list of such symbols as a \emph{sentence}.
A special form of a sentence is the \emph{empty sentence}, written $\emptysentence$.

As commonly done, we aggregate all derivations associated with the same non-terminal symbol
by writing $N \derives R_1 \;|\; \ldots \;|\; R_k$.
We refer to the right-hand-side sentences $R_{1..k}$ as the \emph{alternatives} of $N$.

Our grammar contains another form of derivation --- \emph{inlined derivation} ---
written as $N \derivesinline R_1 \;|\; \ldots \;|\; R_k$.
Expanding an inlined derivation consists of replacing each instance of $N$
in a right-hand-side sentence of a derivation with each of $R_{1..k}$, thereby
creating $k$ variations of it (and removing $N \derivesinline R_1 \;|\; \ldots \;|\; R_k$
from the set of derivations).

For example, consider the derivation
\begin{flalign*}
\Nexpr \derives\ & \Nexpr \parsesep \Nbinop \parsesep \Nexpr &
\end{flalign*}
coupled with the derivation
\begin{flalign*}
\Nbinop \derives\ & \Tand \;|\; \Tband \;|\; \Tbor \;|\; \Tbeq \;|\; \Tdiv \;|\; \Tdivrm \;|\; \Txor \;|\; \Teqop \;|\; \Tneq &\\
                      |\ & \Tgt \;|\; \Tgeq \;|\; \Timpl \;|\; \Tlt \;|\; \Tleq \;|\; \Tplus \;|\; \Tminus \;|\; \Tmod \;|\; \Tmul &\\
                      |\ & \Tor \;|\; \Trdiv \;|\; \Tshl \;|\; \Tshr \;|\; \Tpow \;|\; \Tcoloncolon \;|\; \Tconcat
\end{flalign*}

A grammar containing these two derivations results in shift-reduce conflicts.
Resolving these conflicts is done by associating priority levels to each of the binary operators
and creating a version of the first derivation for each binary operator:
\begin{flalign*}
\Nexpr \derives\ & \Nexpr \parsesep \Tand \parsesep \Nexpr & \\
              |\ & \Nexpr \parsesep \Tband \parsesep \Nexpr & \\
              |\ & \Nexpr \parsesep \Tbor \parsesep \Nexpr & \\
              \ldots \\
              |\ & \Nexpr \parsesep \Tconcat \parsesep \Nexpr &
\end{flalign*}

By defining the derivations of $\Nbinop$ as inlined, we achieve the same effect more compactly:
\begin{flalign*}
\Nbinop \derivesinline\ & \Tand \;|\; \Tband \;|\; \Tbor \;|\; \Tbeq \;|\; \Tdiv \;|\; \Tdivrm \;|\; \Txor \;|\; \Teqop \;|\; \Tneq &\\
                      |\ & \Tgt \;|\; \Tgeq \;|\; \Timpl \;|\; \Tlt \;|\; \Tleq \;|\; \Tplus \;|\; \Tminus \;|\; \Tmod \;|\; \Tmul &\\
                      |\ & \Tor \;|\; \Trdiv \;|\; \Tshl \;|\; \Tshr \;|\; \Tpow \;|\; \Tcoloncolon  \;|\; \Tconcat
\end{flalign*}

Barring mutually-recursive derivations involving inlined derivations, it is possible to expand
all inlined derivations to obtain a context-free grammar without any inlined derivations.

\section{Parametric Productions \label{sec:ParametricProductions}}
A parametric production has the form
$N(p_{1..m}) \derives R_1 \;|\; \ldots \;|\; R_k$
where $p_{1..m}$ are place holders for grammar symbols and may appear in any of the alternatives $R_{1..k}$.
We refer to $N(p_{1..m})$ as a \emph{parametric non-terminal}.

\newcommand\uniquesymb[1]{\textsf{unique}(#1)}
Given sentences $S_{1..m}$, we can expand $N(p_{1..m}) \derives R_1 \;|\; \ldots \;|\; R_k$
by creating a unique symbols for $N(p_{1..m})$, denoted as $\uniquesymb{N(S_{1..m})}$, defining the
derivations
\[
  \uniquesymb{N(S_{1..m})} \derives R_1[S_1/p_1,\ldots,S_m/p_m] \;|\; \ldots \;|\; R_k[S_1/p_1,\ldots,S_m/p_m]
\]
where for each $i= 1..k$, $R_i[S_1/p_1,\ldots,S_m/p_m]$ means replacing each instance of $p_j$ with $S_j$, for each $j=1..m$.
Then, each instance of $S_{1..m}$ in the grammar is replaced by $\uniquesymb{N(S_{1..m})}$.
If all instances of a parametric non-terminal are expanded this way, we can remove the derivations of the parametric
non-terminal altogether.

We note that a parametric production can be either a normal derivation or an inlined derivation.

For example, the derivation for a list of ASL global declarations is as follows:
\begin{flalign*}
\Nast \derives\ & \maybeemptylist{\Ndecl} &
\end{flalign*}
It is defined via the parametric production for possibly-empty lists:
\begin{flalign*}
\maybeemptylist{x}   \derives\ & \emptysentence \;|\; x \parsesep \maybeemptylist{x} &\\
\end{flalign*}

\newcommand\Ndecllist[0]{\nonterminal{decl\_list}}
Expanding $\maybeemptylist{\Ndecl}$ produces the following derivations for a new unique symbol.
That is, a symbol that does not appear anywhere else in the grammar.
In this example we will choose $\uniquesymb{\maybeemptylist{\Ndecl}}$ the be the symbol $\Ndecllist$.
The result of the expansion is then:
\begin{flalign*}
\Ndecllist   \derives\ & \emptysentence \;|\; \Ndecl \parsesep \Ndecllist &\\
\end{flalign*}
The new symbol is substituted anywhere $\maybeemptylist{\Ndecl}$ appears in the original grammar,
which results in the following derivation replacing the original derivation for $\Nast$:
\begin{flalign*}
\Nast \derives\ & \Ndecllist &
\end{flalign*}

% For example,
% \[
%   \option{x} \derives \emptysentence \;|\; x
% \]
% is useful for compactly defining derivations where part of a sentence may or may not appear.

% Suppose we have
% \[
% B \derives a \parsesep \option{A \parsesep b}
% \]
% then expanding the instance $\option{A, b}$ produces
% \[
% \uniquesymb{\option{A \parsesep b}} \derives \emptysentence \;|\; A \parsesep b
% \]
% since $x[A \parsesep b/x]$ yields $A \parsesep b$ and $\emptysentence[A \parsesep b/x]$ yields $\emptysentence$,
% and the derivations for $B$ are replaced by
% \[
% B \derives a \parsesep \uniquesymb{\option{A \parsesep b}}
% \]

Expanding all instances of parametric productions results in a grammar without any parametric productions.

\section{ASL Parametric Productions \label{sec:ASLParametricProductions}}
We define the following parametric productions for various types of lists and optional productions.

\paragraph{Optional Symbol}
\hypertarget{def-option}{}
\begin{flalign*}
\option{x}   \derives\ & \emptysentence \;|\; x &\\
\end{flalign*}

\paragraph{Possibly-empty List}
\hypertarget{def-maybeemptylist}{}
\begin{flalign*}
\maybeemptylist{x}   \derives\ & \emptysentence \;|\; x \parsesep \maybeemptylist{x} &\\
\end{flalign*}

\paragraph{Non-empty List}
\hypertarget{def-nonemptylist}{}
\begin{flalign*}
\nonemptylist{x}   \derives\ & x \;|\; x \parsesep \nonemptylist{x}&\\
\end{flalign*}

\paragraph{Non-empty Comma-separated List}
\hypertarget{def-nclist}{}
\begin{flalign*}
\NClist{x}   \derives\ & x \;|\; x \parsesep \Tcomma \parsesep \NClist{x} &\\
\end{flalign*}

\paragraph{Possibly-empty Comma-separated List}
\hypertarget{def-clist}{}
\begin{flalign*}
\Clist{x}   \derivesinline\ & \emptysentence \;|\; \NClist{x} &\\
\end{flalign*}

\paragraph{Comma-separated List With At Least Two Elements}
\hypertarget{def-clisttwo}{}
\begin{flalign*}
\Clisttwo{x}   \derivesinline\ & x \parsesep \Tcomma \parsesep \NClist{x} &\\
\end{flalign*}

\paragraph{Possibly-empty Parenthesized, Comma-separated List}
\hypertarget{def-plist}{}
\begin{flalign*}
\Plist{x}   \derivesinline\ & \Tlpar \parsesep \Clist{x} \parsesep \Trpar &\\
\end{flalign*}

\paragraph{Parenthesized Comma-separated List With At Least Two Elements}
\hypertarget{def-plisttwo}{}
\begin{flalign*}
\Plisttwo{x}   \derivesinline\ & \Tlpar \parsesep x \parsesep \Tcomma \parsesep \NClist{x} \parsesep \Trpar &\\
\end{flalign*}

\paragraph{Non-empty Comma-separated Trailing List}
\hypertarget{def-tclist}{}
\begin{flalign*}
\NTClist{x}   \derives\ & x \parsesep \option{\Tcomma} &\\
                          |\  & x \parsesep \Tcomma \parsesep \NTClist{x}
\end{flalign*}

\paragraph{Comma-separated Trailing List}
\hypertarget{def-tclist}{}
\begin{flalign*}
\TClist{x}   \derivesinline\ & \option{\NTClist{x}} &\\
\end{flalign*}

\section{ASL Grammar\label{sec:ASLGrammar}}
We now present the list of derivations for the ASL Grammar where the start non-terminal is $\Nast$.
%
The derivations allow certain parse trees where lists may have invalid sizes.
Those parse trees must be rejected in a later phase.

Notice that two of the derivations (for $\Nexprpattern$ and for $\Nexpr$) end with \\
$\precedence{\Tunops}$.
This is a precedence annotation, which is not part of the right-hand-side sentence, and is explained in \secref{PriorityAndAssociativity}
and can be ignored upon first reading.

For brevity, tokens are presented via their label only, dropping their associated value.
For example, instead of $\Tidentifier(\id)$, we simply write $\Tidentifier$.

\hypertarget{def-nast}{}
\begin{flalign*}
\Nast   \derives\ & \productionname{ast}{ast}\ \maybeemptylist{\Ndecl} &
\end{flalign*}

\hypertarget{def-ndecl}{}
\hypertarget{def-funcdecl}{}
\begin{flalign*}
\Ndecl  \derivesinline\ & \productionname{funcdecl}{func\_decl}\ \Tfunc \parsesep \Tidentifier \parsesep \Nparamsopt \parsesep \Nfuncargs \parsesep \Nreturntype \parsesep \Nfuncbody &
\hypertarget{def-proceduredecl}{}\\
|\ & \productionname{proceduredecl}{procedure\_decl}\ \Tfunc \parsesep \Tidentifier \parsesep \Nparamsopt \parsesep \Nfuncargs \parsesep \Nfuncbody &
\hypertarget{def-getter}{}\\
|\ & \productionname{getter}{getter}\ \Tgetter \parsesep \Tidentifier \parsesep \Nparamsopt \parsesep \Naccessargs \parsesep \Nreturntype \parsesep \Nfuncbody&
\hypertarget{def-noarggetter}{}\\
|\ & \productionname{noarggetter}{no\_arg\_getter}\ \Tgetter \parsesep \Tidentifier \parsesep \Nreturntype \parsesep \Nfuncbody &
\hypertarget{def-setter}{}\\
|\ & \productionname{setter}{setter}\ \Tsetter \parsesep \Tidentifier \parsesep \Nparamsopt \parsesep \Naccessargs \parsesep \Teq \parsesep \Ntypedidentifier & \\
   & \wrappedline\ \parsesep \Nfuncbody &
\hypertarget{def-noargsetter}{}\\
|\ & \productionname{noargsetter}{no\_arg\_setter}\ \Tsetter \parsesep \Tidentifier \parsesep \Teq \parsesep \Ntypedidentifier \parsesep \Nfuncbody&
\hypertarget{def-typedecl}{}\\
|\ & \productionname{typedecl}{type\_decl}\ \Ttype \parsesep \Tidentifier \parsesep \Tof \parsesep \Ntydecl \parsesep \Nsubtypeopt \parsesep \Tsemicolon&
\hypertarget{def-subtypedecl}{}\\
|\ & \productionname{subtypedecl}{subtype\_decl}\ \Ttype \parsesep \Tidentifier \parsesep \Nsubtype \parsesep \Tsemicolon&
\hypertarget{def-globalstorage}{}\\
|\ & \productionname{globalstorage}{global\_storage}\ \Nstoragekeyword \parsesep \Nignoredoridentifier \parsesep \option{\Tcolon \parsesep \Nty} \parsesep \Teq \parsesep &\\
   & \wrappedline\ \Nexpr \parsesep \Tsemicolon &
\hypertarget{def-globaluninitvar}{}\\
|\ & \productionname{globaluninitvar}{global\_uninit\_var}\ \Tvar \parsesep \Nignoredoridentifier \parsesep \Tcolon \parsesep \Nty \parsesep \Tsemicolon&
\hypertarget{def-globalpragma}{}\\
|\ & \productionname{globalpragma}{global\_pragma}\ \Tpragma \parsesep \Tidentifier \parsesep \Clist{\Nexpr} \parsesep \Tsemicolon&
\end{flalign*}

\hypertarget{def-nsubtype}{}
\begin{flalign*}
\Nsubtype \derivesinline\ & \Tsubtypes \parsesep \Tidentifier \parsesep \Twith \parsesep \Nfields &\\
          |\              & \Tsubtypes \parsesep \Tidentifier &\\
\end{flalign*}

\hypertarget{def-nsubtypeopt}{}
\begin{flalign*}
\Nsubtypeopt           \derivesinline\ & \option{\Nsubtype} &
\end{flalign*}

\hypertarget{def-ntypedidentifier}{}
\begin{flalign*}
\Ntypedidentifier \derivesinline\ & \Tidentifier \parsesep \Nasty &
\end{flalign*}

\hypertarget{def-nopttypeidentifier}{}
\begin{flalign*}
\Nopttypedidentifier \derivesinline\ & \Tidentifier \parsesep \option{\Nasty} &
\end{flalign*}

\hypertarget{def-nasty}{}
\begin{flalign*}
\Nasty \derivesinline\ & \Tcolon \parsesep \Nty &
\end{flalign*}

\hypertarget{def-nreturntype}{}
\begin{flalign*}
\Nreturntype        \derivesinline\ & \Tarrow \parsesep \Nty &
\end{flalign*}

\hypertarget{def-nparamsopt}{}
\begin{flalign*}
\Nparamsopt \derivesinline\ & \emptysentence &\\
                   |\ & \Tlbrace \parsesep \Clist{\Nopttypedidentifier} \parsesep \Trbrace &
\end{flalign*}

\hypertarget{def-naccessargs}{}
\begin{flalign*}
\Naccessargs        \derivesinline\ & \Tlbracket \parsesep \Clist{\Ntypedidentifier} \parsesep \Trbracket &
\end{flalign*}

\hypertarget{def-nfuncargs}{}
\begin{flalign*}
\Nfuncargs          \derivesinline\ & \Tlpar \parsesep \Clist{\Ntypedidentifier} \parsesep \Trpar &
\end{flalign*}

\hypertarget{def-nmaybeemptystmtlist}{}
\begin{flalign*}
\Nmaybeemptystmtlist          \derivesinline\ & \emptysentence \;|\; \Nstmtlist &
\end{flalign*}

\hypertarget{def-nfuncbody}{}
\begin{flalign*}
\Nfuncbody          \derivesinline\ & \Tbegin \parsesep \Nmaybeemptystmtlist \parsesep \Tend &
\end{flalign*}

\hypertarget{def-nignoredoridentifier}{}
\begin{flalign*}
\Nignoredoridentifier \derivesinline\ & \Tminus \;|\; \Tidentifier &
\end{flalign*}

\vspace*{-\baselineskip}
\paragraph{Parsing note:} $\Tvar$ is not derived by $\Nlocaldeclkeyword$ to avoid an LR(1) conflict.
\hypertarget{def-nlocaldeclkeyword}{}
\begin{flalign*}
\Nlocaldeclkeyword \derivesinline\ & \Tlet \;|\; \Tconstant&
\end{flalign*}

\hypertarget{def-nstoragekeyword}{}
\begin{flalign*}
\Nstoragekeyword \derivesinline\ & \Tlet \;|\; \Tconstant \;|\; \Tvar \;|\; \Tconfig&
\end{flalign*}

\hypertarget{def-ndirection}{}
\begin{flalign*}
\Ndirection \derivesinline\ & \Tto \;|\; \Tdownto &
\end{flalign*}

\hypertarget{def-nalt}{}
\begin{flalign*}
\Nalt       \derivesinline\ & \Twhen \parsesep \Npatternlist \parsesep \option{\Twhere \parsesep \Nexpr} \parsesep \Tarrow \parsesep \Nstmtlist &\\
|\ & \Totherwise \parsesep \Nstmtlist &
\end{flalign*}

\hypertarget{def-notherwiseopt}{}
\begin{flalign*}
\Notherwiseopt     \derives\ & \option{\Totherwise \parsesep \Tarrow \parsesep \Nstmtlist} &
\end{flalign*}

\hypertarget{def-ncatcher}{}
\begin{flalign*}
\Ncatcher \derivesinline\ & \Twhen \parsesep \Tidentifier \parsesep \Tcolon \parsesep \Nty \parsesep \Tarrow \parsesep \Nstmtlist &\\
          |\              & \Twhen \parsesep \Nty \parsesep \Tarrow \parsesep \Nstmtlist &\\
\end{flalign*}

\hypertarget{def-nstmt}{}
\begin{flalign*}
\Nstmt \derivesinline\ & \Tif \parsesep \Nexpr \parsesep \Tthen \parsesep \Nstmtlist \parsesep \Nselse \parsesep \Tend &\\
|\ & \Tcase \parsesep \Nexpr \parsesep \Tof \parsesep \nonemptylist{\Nalt} \parsesep \Tend &\\
|\ & \Twhile \parsesep \Nexpr \parsesep \Tdo \parsesep \Nstmtlist \parsesep \Tend &\\
|\ & \Tlooplimit \parsesep \Tlpar \parsesep \Nexpr \parsesep \Trpar \parsesep \Twhile \parsesep \Nexpr \parsesep \Tdo \parsesep \Nstmtlist \parsesep \Tend &\\
|\ & \Tfor \parsesep \Tidentifier \parsesep \Teq \parsesep \Nexpr \parsesep \Ndirection \parsesep
                    \Nexpr \parsesep \Tdo \parsesep \Nstmtlist \parsesep \Tend &\\
|\ & \Ttry \parsesep \Nstmtlist \parsesep \Tcatch \parsesep \nonemptylist{\Ncatcher} \parsesep \Notherwiseopt \parsesep \Tend &\\
|\ & \Tpass \parsesep \Tsemicolon &\\
|\ & \Treturn \parsesep \option{\Nexpr} \parsesep \Tsemicolon &\\
|\ & \Tidentifier \parsesep \Plist{\Nexpr} \parsesep \Tsemicolon &\\
|\ & \Tassert \parsesep \Nexpr \parsesep \Tsemicolon &\\
|\ & \Nlocaldeclkeyword \parsesep \Ndeclitem \parsesep \Teq \parsesep \Nexpr \parsesep \Tsemicolon &\\
|\ & \Nlexpr \parsesep \Teq \parsesep \Nexpr \parsesep \Tsemicolon &\\
|\ & \Tvar \parsesep \Ndeclitem \parsesep \option{\Teq \parsesep \Nexpr} \parsesep \Tsemicolon &\\
|\ & \Tvar \parsesep \Clisttwo{\Tidentifier} \parsesep \Tcolon \parsesep \Nty \parsesep \Tsemicolon &\\
|\ & \Tprint \parsesep \Plist{\Nexpr} \parsesep \Tsemicolon &\\
|\ & \Trepeat \parsesep \Nstmtlist \parsesep \Tuntil \parsesep \Nexpr \parsesep \Tsemicolon &\\
|\ & \Tlooplimit \parsesep \Tlpar \parsesep \Nexpr \parsesep \Trpar \parsesep \Trepeat \parsesep \Nstmtlist \parsesep \Tuntil \parsesep \Nexpr \parsesep \Tsemicolon &\\
|\ & \Tthrow \parsesep \Nexpr \parsesep \Tsemicolon &\\
|\ & \Tthrow \parsesep \Tsemicolon &\\
|\ & \Tpragma \parsesep \Tidentifier \parsesep \Clist{\Nexpr} \parsesep \Tsemicolon &
\end{flalign*}

\hypertarget{def-nstmtlist}{}
\begin{flalign*}
\Nstmtlist \derivesinline\ & \nonemptylist{\Nstmt} &
\end{flalign*}

\hypertarget{def-nselse}{}
\begin{flalign*}
\Nselse \derives\ & \Telseif \parsesep \Nexpr \parsesep \Tthen \parsesep \Nstmtlist \parsesep \Nselse &\\
|\ & \Tpass &\\
|\ & \Telse \parsesep \Nstmtlist &
\end{flalign*}

\hypertarget{def-nlexpr}{}
\begin{flalign*}
\Nlexpr \derivesinline\ & \Nlexpratom &\\
|\ & \Tminus &\\
|\ & \Tlpar \parsesep \NClist{\Nlexpr} \parsesep \Trpar &
\end{flalign*}

\hypertarget{def-nlexpratom}{}
\begin{flalign*}
\Nlexpratom \derives\ & \Tidentifier &\\
|\ & \Nlexpratom \parsesep \Nslices &\\
|\ & \Nlexpratom \parsesep \Tdot \parsesep \Tidentifier &\\
|\ & \Nlexpratom \parsesep \Tdot \parsesep \Tlpar \parsesep \NClist{\Tidentifier} \parsesep \Trpar &\\
|\ & \Tlbracket \parsesep \NClist{{\Nlexpratom}} \parsesep \Trbracket &
\end{flalign*}

A $\Ndeclitem$ is another kind of left-hand-side expression,
which appears only in declarations. It cannot have setter calls or set record fields,
it must declare a new variable.
\hypertarget{def-ndeclitem}{}
\begin{flalign*}
\Ndeclitem \derives\ & \Nuntypeddeclitem \parsesep \Nasty&\\
|\ & \Nuntypeddeclitem  &
\end{flalign*}

\hypertarget{def-nuntypeddeclitem}{}
\begin{flalign*}
\Nuntypeddeclitem \derivesinline\ & \Tidentifier &\\
|\ & \Tminus &\\
|\ & \Plisttwo{\Ndeclitem} &
\end{flalign*}

\hypertarget{def-nintconstraintsopt}{}
\begin{flalign*}
\Nintconstraintsopt \derivesinline\ & \Nintconstraints \;|\; \emptysentence &
\end{flalign*}

\hypertarget{def-nintconstraints}{}
\begin{flalign*}
\Nintconstraints \derivesinline\ & \Tlbrace \parsesep \NClist{\Nintconstraint} \parsesep \Trbrace &
\end{flalign*}

\hypertarget{def-nintconstraint}{}
\begin{flalign*}
\Nintconstraint \derivesinline\ & \Nexpr &\\
|\ & \Nexpr \parsesep \Tslicing \parsesep \Nexpr &
\end{flalign*}

Pattern expressions ($\Nexprpattern$), given by the following derivations, is similar to regular expressions  ($\Nexpr$),
except they do not derive tuples, which are the last derivation for $\Nexpr$.

\hypertarget{def-nexprpattern}{}
\begin{flalign*}
\Nexprpattern \derives\ & \Nvalue &\\
                    |\  & \Tidentifier &\\
                    |\  & \Nexprpattern \parsesep \Nbinop \parsesep \Nexpr &\\
                    |\  & \Nunop \parsesep \Nexpr & \precedence{\Tunops}\\
                    |\  & \Tif \parsesep \Nexpr \parsesep \Tthen \parsesep \Nexpr \parsesep \Neelse &\\
                    |\  & \Tidentifier \parsesep \Plist{\Nexpr} &\\
                    |\  & \Nexprpattern \parsesep \Nslices &\\
                    |\  & \Nexprpattern \parsesep \Tdot \parsesep \Tlpar \parsesep \NClist{\Tidentifier} \parsesep \Trpar &\\
                    |\  & \Nexprpattern \parsesep \Tas \parsesep \Nty &\\
                    |\  & \Nexprpattern \parsesep \Tas \parsesep \Nintconstraints &\\
                    |\  & \Nexprpattern \parsesep \Tin \parsesep \Npatternset &\\
                    |\  & \Nexprpattern \parsesep \Tin \parsesep \Tmasklit &\\
                    |\  & \Tunknown \parsesep \Tcolon \parsesep \Nty &\\
                    |\  & \Tidentifier \parsesep \Tlbrace \parsesep \Clist{\Nfieldassign} \parsesep \Trbrace &\\
                    |\  & \Tlpar \parsesep \Nexprpattern \parsesep \Trpar &
\end{flalign*}

\hypertarget{def-npatternset}{}
\begin{flalign*}
\Npatternset \derivesinline\  & \Tbnot \parsesep \Tlbrace \parsesep \Npatternlist \parsesep \Trbrace &\\
                  |\    & \Tlbrace \parsesep \Npatternlist \parsesep \Trbrace &
\end{flalign*}

\hypertarget{def-npatternlist}{}
\begin{flalign*}
\Npatternlist \derivesinline\ & \NClist{\Npattern} &
\end{flalign*}

\hypertarget{def-npattern}{}
\begin{flalign*}
\Npattern \derives\ & \Nexprpattern &\\
                |\  & \Nexprpattern \parsesep \Tslicing \parsesep \Nexpr &\\
                |\  & \Tminus &\\
                |\  & \Tleq \parsesep \Nexpr &\\
                |\  & \Tgeq \parsesep \Nexpr &\\
                |\  & \Tmasklit &\\
                |\  & \Plisttwo{\Npattern} &\\
                |\  & \Npatternset &
\end{flalign*}

\hypertarget{def-nfields}{}
\begin{flalign*}
\Nfields \derivesinline\ & \Tlbrace \parsesep \TClist{\Ntypedidentifier} \parsesep \Trbrace &
\end{flalign*}

\hypertarget{def-nfieldsopt}{}
\begin{flalign*}
\Nfieldsopt \derivesinline\ & \Nfields \;|\; \emptysentence &
\end{flalign*}

\hypertarget{def-nnslices}{}
\begin{flalign*}
\Nnamedslices \derivesinline\ & \Tlbracket \parsesep \NClist{\Nslice} \parsesep \Trbracket &
\end{flalign*}

\hypertarget{def-nslices}{}
\begin{flalign*}
\Nslices \derivesinline\ & \Tlbracket \parsesep \Clist{\Nslice} \parsesep \Trbracket &
\end{flalign*}

\hypertarget{def-nslice}{}
\begin{flalign*}
\Nslice \derivesinline\ & \Nexpr &\\
              |\  & \Nexpr \parsesep \Tcolon \parsesep \Nexpr &\\
              |\  & \Nexpr \parsesep \Tpluscolon \parsesep \Nexpr &\\
              |\  & \Nexpr \parsesep \Tstarcolon \parsesep \Nexpr &
\end{flalign*}

\hypertarget{def-nbitfields}{}
\begin{flalign*}
\Nbitfields \derivesinline\ & \Tlbrace \parsesep \TClist{\Nbitfield} \parsesep \Trbrace &
\end{flalign*}

\hypertarget{def-nbitfield}{}
\begin{flalign*}
\Nbitfield \derivesinline\ & \Nnamedslices \parsesep \Tidentifier &\\
                  |\ & \Nnamedslices \parsesep \Tidentifier \parsesep \Nbitfields &\\
                  |\ & \Nnamedslices \parsesep \Tidentifier \parsesep \Tcolon \parsesep \Nty &
\end{flalign*}

\hypertarget{def-nty}{}
\begin{flalign*}
\Nty \derives\ & \Tinteger \parsesep \Nintconstraintsopt &\\
            |\ & \Treal &\\
            |\ & \Tstring &\\
            |\ & \Tboolean &\\
            |\ & \Tbit &\\
            |\ & \Tbits \parsesep \Tlpar \parsesep \Nexpr \parsesep \Trpar \parsesep \maybeemptylist{\Nbitfields} &\\
            |\ & \Plist{\Nty} &\\
            |\ & \Tidentifier &\\
            |\ & \Tarray \parsesep \Tlbracket \parsesep \Nexpr \parsesep \Trbracket \parsesep \Tof \parsesep \Nty &
\end{flalign*}

\hypertarget{def-ntydecl}{}
\begin{flalign*}
\Ntydecl \derives\ & \Nty &\\
            |\ & \Tenumeration \parsesep \Tlbrace \parsesep \NTClist{\Tidentifier} \parsesep \Trbrace &\\
            |\ & \Trecord \parsesep \Nfieldsopt &\\
            |\ & \Texception \parsesep \Nfieldsopt &
\end{flalign*}

\hypertarget{def-nfieldassign}{}
\begin{flalign*}
\Nfieldassign \derivesinline\ & \Tidentifier \parsesep \Teq \parsesep \Nexpr &
\end{flalign*}

\hypertarget{def-neelse}{}
\begin{flalign*}
\Neelse \derives\ & \Telse \parsesep \Nexpr &\\
                     |\ & \Telseif \parsesep \Nexpr \parsesep \Tthen \parsesep \Nexpr \parsesep \Neelse &
\end{flalign*}

\hypertarget{def-nexpr}{}
\begin{flalign*}
\Nexpr \derives\  & \Nvalue &\\
                    |\  & \Tidentifier &\\
                    |\  & \Nexpr \parsesep \Nbinop \parsesep \Nexpr &\\
                    |\  & \Nunop \parsesep \Nexpr & \precedence{\Tunops}\\
                    |\  & \Tif \parsesep \Nexpr \parsesep \Tthen \parsesep \Nexpr \parsesep \Neelse &\\
                    |\  & \Tidentifier \parsesep \Plist{\Nexpr} &\\
                    |\  & \Nexpr \parsesep \Nslices &\\
                    |\  & \Nexpr \parsesep \Tdot \parsesep \Tidentifier&\\
                    |\  & \Nexpr \parsesep \Tdot \parsesep \Tlbracket \parsesep \NClist{\Tidentifier} \parsesep \Trbracket &\\
                    |\  & \Tlbracket \parsesep \NClist{\Nexpr} \parsesep \Trbracket &\\
                    |\  & \Nexpr \parsesep \Tas \parsesep \Nty &\\
                    |\  & \Nexpr \parsesep \Tas \parsesep \Nintconstraints &\\
                    |\  & \Nexpr \parsesep \Tin \parsesep \Npatternset &\\
                    |\  & \Nexpr \parsesep \Tin \parsesep \Tmasklit &\\
                    |\  & \Tunknown \parsesep \Tcolon \parsesep \Nty &\\
                    |\  & \Tidentifier \parsesep \Tlbrace \parsesep \Clist{\Nfieldassign} \parsesep \Trbrace &\\
                    |\  & \Tlpar \parsesep \Nexpr \parsesep \Trpar &\\
                    |\  & \Plisttwo{\Nexpr} &
\end{flalign*}

\hypertarget{def-nvalue}{}
\begin{flalign*}
\Nvalue \derivesinline\ & \Tintlit &\\
                     |\ & \Tboollit &\\
                     |\ & \Treallit &\\
                     |\ & \Tbitvectorlit &\\
                     |\ & \Tstringlit &
\end{flalign*}

\hypertarget{def-nunop}{}
\begin{flalign*}
\Nunop \derivesinline\ & \Tbnot \;|\; \Tminus \;|\; \Tnot &
\end{flalign*}

\hypertarget{def-nbinop}{}
\begin{flalign*}
\Nbinop \derivesinline\ & \Tand \;|\; \Tband \;|\; \Tbor \;|\; \Tbeq \;|\; \Tdiv \;|\; \Tdivrm \;|\; \Txor \;|\; \Teqop \;|\; \Tneq &\\
                     |\ & \Tgt \;|\; \Tgeq \;|\; \Timpl \;|\; \Tlt \;|\; \Tleq \;|\; \Tplus \;|\; \Tminus \;|\; \Tmod \;|\; \Tmul &\\
                     |\ & \Tor \;|\; \Trdiv \;|\; \Tshl \;|\; \Tshr \;|\; \Tpow \;|\; \Tconcat
\end{flalign*}

\section{Parse Trees \label{sec:ParseTrees}}
We now define \emph{parse trees} for the ASL expanded grammar. Those are later used for build Abstract Syntax Trees.

\begin{definition}[Parse Trees]
A \emph{parse tree} has one of the following forms:
\begin{itemize}
  \item A \emph{token node}, given by the token itself, for example, $\Tlexeme(\Tarrow)$ and $\Tidentifier(\id)$;
  \item \hypertarget{def-epsilonnode}{} $\epsilonnode$, which represents the empty sentence --- $\emptysentence$.
  \item A \emph{non-terminal node} of the form $N(n_{1..k})$ where $N$ is a non-terminal symbol,
        which is said to label the node,
        and $n_{1..k}$ are its children parse nodes,
        for example,
        $\Ndecl(\Tfunc, \Tidentifier(\id), \Nparamsopt, \Nfuncargs, \Nfuncbody)$
        is labeled by $\Ndecl$ and has five children nodes.
\end{itemize}
\end{definition}
(In the literature, parse trees are also referred to as \emph{derivation trees}.)

\begin{definition}[Well-formed Parse Trees]
A parse tree is \emph{well-formed} if its root is labelled by the start non-terminal ($\Nast$ for ASL)
and each non-terminal node $N(n_{1..k})$ corresponds to a grammar derivation
$N \derives l_{1..k}$ where $l_i$ is the label of node $n_i$ if it is a non-terminal node and $n_i$
itself when it is a token.
A non-terminal node $N(\epsilonnode)$ is well-formed if the grammar includes a derivation
$N \derives \emptysentence$.
\end{definition}

\hypertarget{def-yield}{}
\begin{definition}[Parse Tree Yield]
The \emph{\yield} of a parse tree is the list of tokens
given by an in-order walk of the tree:
\[
\yield(n) \triangleq \begin{cases}
  [t] & n \text{ is a token }t\\
  \emptylist & n = \epsilonnode\\
  \yield(n_1) \concat \ldots \concat \yield(n_k) & n = N(n_{1..k})\\
\end{cases}
\]
\end{definition}

\hypertarget{def-parsenode}{}
We denote the set of well-formed parse trees for a non-terminal symbol $S$ by $\parsenode{S}$.

\hypertarget{def-aslparse}{}
A parser is a function
\[
\aslparse : (\Token^* \setminus \{\Terror\}) \aslto \parsenode{\Nast} \cup \{\ParseError\}
\]
\hypertarget{def-parseerror}{}
where $\ParseError$ stands for a \emph{parse error}.
If $\aslparse(\ts) = n$ then $\yield(n)=\ts$
and if $\aslparse(\ts) = \ParseError$ then there is no well-formed tree
$n$ such that $\yield(n)=\ts$.
(Notice that we do not define a parser if $\ts$ is lexically illegal.)

The \emph{language of a grammar} $G$ is defined as follows:
\[
\Lang(G) = \{\yield(n) \;|\; n \text{ is a well-formed parse tree for }G\} \enspace.
\]

\section{Priority and Associativity \label{sec:PriorityAndAssociativity}}
A context-free grammar $G$ is \emph{ambiguous} if there can be more than one parse tree for a given list of tokens
$\ts \in \Lang(G)$.
Indeed the expanded ASL grammar is ambiguous, for example, due to its definition of binary operation expressions.
To allow assigning a unique parse tree to each sequence of tokens in the language of the ASL grammar,
we utilize the standard technique of associating priority levels to productions and using them to resolve
any shift-reduce conflicts in the LR(1) parser associated with our grammar (our grammar does not have any
reduce-reduce conflicts).

The priority of a grammar derivation is defined as the priority of its rightmost token.
Derivations that do not contain tokens do not require a priority as they do not induce shift-reduce conflicts.

The table below assigns priorities to tokens in increasing order, starting from the lowest priority (for $\Telse$)
to the highest priority (for $\Tdot$).
When a shift-reduce conflict arises during the LR(1) grammar construction
it resolve in favor of the action (shift or reduce) associated with the derivation that has the higher priority.
If two derivations have the same priority due to them both having the same rightmost token,
the conflict is resolved based on the associativity associated with the token below:
reduce for $\leftassoc$, shift for $\rightassoc$, and a parsing error for $\nonassoc$.

The two rules involving a unary minus operation are not assigned the priority level of $\Tminus$,
but rather then the priority level $\Tunops$, as denoted by the notation \\
$\precedence{\Tunops}$
appearing to their right. This is a standard way of dealing with a unary minus operation
in many programming languages, which involves defining an artificial token $\Tunops$,
which is never returned by the scanner.

\begin{center}
\begin{tabular}{ll}
\textbf{Terminals} & \textbf{Associativity}\\
\hline
\Telse & \nonassoc\\
\Tbor, \Tband, \Timpl, \Tbeq, \Tas & \leftassoc\\
\Teqop, \Tneq & \leftassoc\\
\Tgt, \Tgeq, \Tlt, \Tleq & \nonassoc\\
\Tplus, \Tminus, \Tor, \Txor, \Tand & \leftassoc\\
\Tmul, \Tdiv, \Tdivrm, \Trdiv, \Tmod, \Tshl, \Tshr & \leftassoc\\
\Tpow, \Tconcat & \leftassoc\\
$\Tunops$ & \nonassoc\\
\Tin & \nonassoc\\
\Tdot, \Tlbracket & \leftassoc
\end{tabular}
\end{center}
